\documentclass[12pt,letterpaper]{article}

\usepackage{amsmath}	% just math
\usepackage{amssymb}	% allow blackboard bold (aka N,R,Q sets)
\linespread{1.6}	% double spaces lines

\textwidth 6.5truein  % These 4 commands define more efficient margins
\textheight 9.5truein
\oddsidemargin 0.0in
\topmargin -0.6in

\parindent 0pt	% let's not indent paragraphs
\parskip 5pt  % Also, a bit of space between paragraphs

\begin{document}
\begin{flushright}
\linespread{1}	% single spaces lines
\small \normalsize %% dumb, but have to do this for the prev to work
Jeremy Osterhouse \\
\today
\end{flushright}

{\bf Problem.} Prove the Pinching Theorem: \emph{Suppose that for all sufficiently large n,}
\[
a_n \leq b_n \leq c_n.
\]
\emph{If $\lim{a_n} = \lim{c_n} = L$, then $\lim{b_n}=L$.}

{\bf Solution.} To prove this, we will use the definition of the limit of a sequence. Because we know that $\lim{a_n} = \lim{c_n} = L$, we know by the definition of the limit of a sequence that,
\[
L-\epsilon_a < a_n < L+\epsilon_a
\]
for any $\epsilon_a$ with all sufficiently large $n$ and that,
\[
L-\epsilon_c < c_n < L+\epsilon_c
\]
for any $\epsilon_c$ with all sufficiently large $n$.

Because we know that $a_n<c_n$, let $\epsilon = max\{\epsilon_a,\epsilon_c\}$. Now we get
\[
L-\epsilon < a_n \leq c_n < L + \epsilon
\]
But we know that $a_n \leq b_n \leq c_n$. For any $\epsilon$, then there exists an $\epsilon_b$ such that for sufficiently large $n$,
\[
L-\epsilon \leq L-\epsilon_b < a_n \leq b_n \leq c_n < L+\epsilon_b \leq L+\epsilon.
\]
Removing the unneccesary terms of this inequality we get,
\[
L-\epsilon < b_n < L+\epsilon
\]
for any $\epsilon$ with all sufficiently large $n$. This is equivalent to saying that $\lim{b_n}=L$.
\hfill$\square$

%add dollar signs around visual selection and leave in insert at end:
%-->`>a$`<i$f$a

%add QED symbol abbreviation
%-->ab qed. \hfill$\square$ 




\end{document}
