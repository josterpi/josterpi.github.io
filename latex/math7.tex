\documentclass[12pt,letterpaper]{article}

\usepackage{amsmath}	% just math
\usepackage{amssymb}	% allow blackboard bold (aka N,R,Q sets)
\usepackage{amsthm}	% allow blackboard bold (aka N,R,Q sets)
\linespread{1.6}	% double spaces lines

\textwidth 6.5truein  % These 4 commands define more efficient margins
\textheight 9.5truein
\oddsidemargin 0.0in
\topmargin -0.6in

\parindent 0pt	% let's not indent paragraphs
\parskip 5pt  % Also, a bit of space between paragraphs

\newtheorem*{lemma1}{Lemma 1}
\newtheorem*{lemma2}{Lemma 2}

\begin{document}
\begin{flushright}
\linespread{1}	% single spaces lines
\small \normalsize %% dumb, but have to do this for the prev to work
Jeremy Osterhouse \\
\today
\end{flushright}

{\bf Problem 1.} Suppose
\begin{enumerate}
\item[a)] $f(x)$ continuous for $x \geq a$.
\item[b)] $f(x) \geq 0$ for $x \geq x$.
\item[c)] $b>a$.
\end{enumerate}
Show that
\[
\int_a^\infty f(x) dx \mbox{ convergant } \iff \int_b^\infty f(x) dx \mbox{ convergent.}
\]
{\bf Solution 1.}
\begin{lemma1} For $b>a$,
\[
\int_a^\infty f(x) dx = \int_a^b f(x) dx + \int_b^\infty f(x) dx.
\]
\begin{proof}
By the definition of an indefinite integral,
\[
\int_a^\infty f(x) dx = \lim_{n \to \infty} \int_a^n f(x) dx.
\]
We can split this up into $\lim \int_a^b f(x)dx + \lim \int_b^n f(x)dx$ because we now have a definite integral. Again using the definition of an indefinite integral we have
\[
\int_a^b f(x)dx + \int_b^\infty f(x)dx = \int_a^\infty f(x) dx.
\]
\end{proof}
\end{lemma1}

Now, since we are trying to show an if and only if relationship, we must prove the implication both ways. First, let $\int_a^\infty f(x)dx$ be convergant. We must show that $\int_b^\infty f(x)dx$ is convergant. By Lemma 1, we have that $\int_a^\infty f(x) dx = \int_a^b f(x) dx + \int_b^\infty f(x) dx.$ It is given that $\int_a^\infty f(x)dx$ is convergant. $\int_a^b f(x)dx$ is simply a definite integral, so we know that it has some constant value. With these two convergant integrals, we can define $\int_b^\infty f(x)\,dx$ in terms of the integrals from $a$ to $\infty$ and $a$ to $b$.
\[
\int_b^\infty f(x)\,dx = \int_a^\infty f(x)\,dx - \int_a^b f(x)\,dx
\]
And so if $\int_a^\infty f(x)\,dx$ is convergant, then $\int_b^\infty f(x)\,dx$ must be convergant.
\hfill$\square$

Next we must show that the converse is true. Let $\int_b^\infty f(x)\,dx$ converge. We must show that this implies that $\int_a^\infty f(x)\,dx$ converges. The argument is very similar to the argument above. Again, by Lemma 1, we have that $\int_a^\infty f(x) dx = \int_a^b f(x) dx + \int_b^\infty f(x) dx$. It is given that $\int_b^\infty f(x)\,dx$ is convergant. $\int_a^b f(x)dx$ is simply a definite integral, so we know that it has some constant value. So, with Lemma 1, we have $\int_b^\infty f(x)\,dx$ equal to two integrals with constant value. Therefore, $\int_b^\infty f(x)\,dx$ must converge.
\hfill$\square$

And so the implication is proven in both directions and the if and only if relationship is true.
\hfill$\square$

{\bf Problem 2.} Suppose that $\{a_n\}$ is a sequence such that $a_n$ is defined for $n \geq k$. Suppose that $l>k$. Show
\[
\sum_{n=k}^\infty a_n \mbox{ is convergant } \iff \sum_{n=l}^\infty a_n \mbox { is convergant.}
\]

{\bf Solution 2.} 
\begin{lemma2} For $l>k$,
\[
\sum_{n=k}^\infty a_n = \sum_{n=k}^l a_n + \sum_{n=l}^\infty a_n.
\]
\begin{proof}
By the definition of an indefinite series,
\[
\sum_{n=k}^\infty a_n = \lim_{n \to \infty} \sum_{n=k}^n a_n.
\]
We can split this up into $\lim \sum_{n=k}^l a_n + \lim \sum_{n=l}^n a_n$ because we now have a definite series. Again using the definition of an indefinite series we have
\[
\sum_{n=k}^l a_n + \sum_{n=l}^\infty a_n = \sum_{n=k}^\infty a_n.
\]
\end{proof}
\end{lemma2}

Now, since we are trying to show an if and only if relationship, we must prove the implication both ways. First, let $\sum_{n=k}^\infty a_n$ be convergant. We must show that $\sum_{n=l}^\infty a_n$ is convergant. By Lemma 2, we have that $\sum_{n=k}^\infty a_n = \sum_{n=k}^l a_n + \sum_{n=l}^\infty a_n.$ It is given that $\sum_{n=k}^\infty a_n$ is convergant. $\sum_{n=k}^l a_n$ is simply a definite series, so we know that it has some constant value. With these two convergant series, we can define $\sum_{n=l}^\infty a_n$ in terms of the series from $k$ to $\infty$ and $k$ to $l$.
\[
\sum_l^\infty a_n = \sum_{n=k}^\infty a_n - \sum_{n=k}^l a_n
\]
And so if $\sum_{n=k}^\infty a_n$ is convergant, then $\sum_{n=l}^\infty a_n$ must be convergant.
\hfill$\square$

Next we must show that the converse is true. Let $\sum_{n=l}^\infty a_n$ converge. We must show that this implies that $\sum_{n=k}^\infty a_n$ converges. The argument is very similar to the argument above. Again, by Lemma 2, we have that $\sum_{n=k}^\infty a_n = \sum_{n=k}^l a_n + \sum_{n=l}^\infty a_n$. It is given that $\sum_{n=l}^\infty a_n$ is convergant. $\sum_{n=k}^l a_n$ is simply a definite series, so we know that it has some constant value. So, with Lemma 2, we have $\sum_{n=l}^\infty a_n$ equal to two series with constant value. Therefore, $\sum_{n=l}^\infty a_n$ must converge.
\hfill$\square$

And so the implication is proven in both directions and the if and only if relationship is true.
\hfill$\square$


%add dollar signs around visual selection and leave in insert at end:
%-->`>a$`<i$f$a

%add QED symbol abbreviation
%-->ab qed. \hfill$\square$ 




\end{document}
